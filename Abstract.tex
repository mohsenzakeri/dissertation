%Abstract Page

\hbox{\ }

\renewcommand{\baselinestretch}{1}
\small \normalsize

\begin{center}
\large{{ABSTRACT}}

\vspace{3em}

\end{center}
\hspace{-.15in}
\begin{tabular}{ll}
Title of Dissertation:    & {\large  OPTIMIZING THE ACCURACY OF }\\
&                     {\large  LIGHTWEIGHT METHODS FOR SHORT READ} \\
&                     {\large  ALIGNMENT AND QUANTIFICATION} \\
\ \\
&                          {\large  Mohsen Zakeri} \\
&                           {\large Doctor of Philosophy, 2021} \\
\ \\
Dissertation Directed by: & {\large  Professor Rob Patro} \\
&               {\large  Department of Computer Science } \\
\end{tabular}

\vspace{3em}

\renewcommand{\baselinestretch}{2}
\large \normalsize

The analysis of the high throughput sequencing (HTS) data includes a number 
of involved computational steps, ranging from the assembly of transcriptome, 
mapping or alignment of the reads to existing or assembled 
sequences, estimating the abundance of sequenced molecules, performing 
differential or comparative analysis between samples, and even inferring 
dynamics of interest from snapshot data. Many methods have been developed 
for these different tasks that provide various trade-offs in terms of 
accuracy and speed, because accuracy and robustness typically come at the 
expense of sacrificing speed and vice versa. In this work, I focus on the 
problems of alignment and quantification of RNA-seq data, and review 
different aspects of the available methods for these problems. I explore 
finding a reasonable balance between these competing goals, and introduce 
methods that provide accurate results without sacrificing speed.

Alignment of sequencing reads to known reference sequences is 
a challenging computational step in the RNA-seq pipeline mainly because of 
the large size of sample data and reference sequences, and highly-repetitive 
sequence. Recently, the concept of lightweight alignment
is introduced to accelerate the mapping step of abundance estimation. 
I collaborated with my colleagues to explore some 
of the shortcomings of the lightweight alignment methods, and to address 
those with a new approach called the selective-alignment. Moreover, we 
introduce an aligner, Puffaligner, which benefits from both the indexing 
approach introduced by the Pufferfish index and also selective-alignment 
to produce accurate alignments in a short amount of time compared to other 
popular aligners.

To improve the speed of RNA-seq quantification given a collection of alignments, 
some tools group fragments (reads) into equivalence classes which are sets of 
fragments that  are compatible with the same subset of reference sequences. 
Summarizing the fragments into equivalence classes factorizes the likelihood 
function being optimized and increases the speed of the typical optimization 
algorithms deployed. I explore how this factorization affects the accuracy of 
abundance estimates, and propose a new factorization approach which demonstrates 
higher fidelity to the non-approximate model.

Finally, estimating the posterior distribution of the transcript expressions is 
a crucial step in finding robust and reliable estimates of transcript abundance 
in the presence of high levels of multi-mapping. To assess the accuracy of their 
point estimates, quantification tools generate inferential replicates using 
techniques such as Bootstrap sampling and Gibbs sampling. The utility of inferential 
replicates has been portrayed in different downstream RNA-seq applications, i.e., 
performing differential expression analysis. I explore how sampling from both 
observed and unobserved data points (reads) improves the accuracy of Bootstrap 
sampling. I demonstrate the utility of this approach in estimating allelic 
expression with RNA-seq reads, where the absence of unique mapping reads to 
reference transcripts is a major obstacle for calculating robust estimates.