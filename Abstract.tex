%Abstract Page

\hbox{\ }

\renewcommand{\baselinestretch}{1}
\small \normalsize

\begin{center}
\large{{ABSTRACT}}

\vspace{3em}

\end{center}
\hspace{-.15in}
\begin{tabular}{ll}
Title of Dissertation:    & {\large  OPTIMIZING THE ACCURACY OF }\\
&                     {\large  LIGHTWEIGHT METHODS FOR SHORT READ} \\
&                     {\large  ALIGNMENT AND QUANTIFICATION} \\
\ \\
&                          {\large  Mohsen Zakeri} \\
&                           {\large Doctor of Philosophy, 2021} \\
\ \\
Dissertation Directed by: & {\large  Professor Rob Patro} \\
&               {\large  Department of Physics } \\
\end{tabular}

\vspace{3em}

\renewcommand{\baselinestretch}{2}
\large \normalsize

The analysis of the high throughput sequencing (HTS) data includes 
a number of involved computational steps, ranging from the assembly 
of the reference sequences, mapping or alignment of the reads to 
existing or assembled sequences, estimating the abundance of 
sequenced molecules, performing differential or comparative 
analysis between samples, and even inferring dynamics of interest 
from snapshot data. Many methods have been developed for these 
different tasks that provide different trade-offs in terms of 
accuracy and speed, because precision typically comes at the 
expense of sacrificing speed and vice versa. Throughout this 
work, I review different aspects of the available methods for 
alignment and quantification steps of RNA-seq data. Furthermore, 
I explore finding a reasonable balance between these competing 
goals to introduce methods which are designed to be almost as 
good as the most accurate approaches, while being as fast as the 
methods that focus on speed.

Alignment or mapping of the sequencing reads to the known reference 
sequences is a challenging computational step in the pipeline because 
of the large size of sample data. A typical RNA-seq sample often consists 
of 10s of millions of paired-end reads which should be queried against the 
large number of reference sequences to find the most similar reference substrings 
under some notion of edit distance. Therefore, the aligners index the reference 
sequences to accelerate the search procedure. Furthermore, recent quantification 
methods, introduced the concept of lightweight alignment in order to accelerate 
the mapping step, and therefore, the whole quantification pipeline. I collabrated 
with my colleagues to explore some of the shortcomings of the lightweight alignments, 
and to try to address those with a new approach called the selective alignment. 
Moreover, we introduce a new aligner, Puffaligner, which benefits from the indexing 
approach introduced by the Pufferfish index and also the idea of selective-alignment 
to produce accurate alignments in a short time compared to other popular aligners.

I have also explored the shortcomings of the approximate generative model used in the 
fast RNA-seq quantifiers. In these methods, fragments (reads) are grouped together 
into equivalence classes which are sets of sequenced fragments for which all the 
fragments are compatible with a specific set of reference sequences. Therefore, in 
the approximate models, all the fragments in each group are treated as identical, 
which factorizes the likelihood function being optimized and increases the speed of 
the optimization step. I have explored how this factorization affects the accuracy of 
abundance estimates, and propose a new factorization approach for approximating the 
likelihood which demonstrates higher fidelity to the exact model.

%Finally, I propose the possible path forward for increasing the accuracy of 
%abundance estimation tools in the cases where there are anomalies in transcript 
%coverages which could lead to the detection of unannoated transcripts. Also, 
%I will investigate if representing single cell expression matrices in terms of 
%equivalence classes, rather than the gene counts, increases the accuracy or 
%robustness of the downstream analysis of the single cell pipelines, such as 
%dimensionality reduction and cell clustering.