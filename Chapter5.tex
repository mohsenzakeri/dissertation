%Chapter 5

\renewcommand{\thechapter}{5}

\chapter[Conclusion]{Conclusion} 
\label{conclusion}

Throughout this dissertation, we have explored improving the lightweight approaches
employed in various steps of the RNA-seq analysis pipeline, i.e., mapping or alignment
of the reads to a known reference, estimating the abundance of transcripts, and 
assessing the accuracy of the point estimates by estimating the posterior distribution.
The recurring theme of all the methods we have introduced here is improving the accuracy
of lightweight methods by maintaining their efficiency.

In~\cref{chapt2}, we introduced selective-alignment as a new algorithm for efficiently
aligning the reads to the reference transcriptome. This approach increases both sensitivity
and specificity of quasi-mapping. Selective-alignment increases the sensitivity by performing
safe skips for querying each \kmer in the reference. It also relaxes other constraints
imposed on merging the mappings discovered for a read. It also introduces the concept of
co-mapping for further refining the candidate mappings. Selective-alignment also computes
an alignment score for each mapping to further filter spurious hits. The alignment-score
could be also used in the quantification step to improve the accuracy of the estimations.
We show how selective-alignment improves the accuracy of the quantification with lightweight
methods without sacrificing the performance.

Furthermore, we have introduced \puffaligner in~\cref{chapt2}. \puffaligner is built on top
of the \pufferfish index which is an efficient \ccdbg base index of a collection of reference
sequences. \puffaligner is a multi-purpose aligner that can be utilized for aligning DNA-seq, 
RNA-seq, and metagenomic reads. \puffaligner finds high-quality alignments for short reads 
which are similar to the ones discovered by accurate tools, e.g., \bt, in a much shorter amount
of time.

We have investigated the effect of the factorization of the likelihood employed
by lightweight RNA-seq quantification tools on the accuracy of the estimations. These tools
treat all the fragments (reads) which are mapped to the same set of reference sequences as identical,
and represent all the fragments compatible with the same set of transcripts as one equivalence
class in the likelihood function. 
This factorization approximates the likelihood function because of the differences in the 
characteristics of each fragment in an equivalence class, e.g., different fragment length
and alignment compatibilities lead to different conditional probabilities.
I have proposed an improved factorization in ~\cref{chapt3} which groups the fragments
in an equivalence class that is not only similar in terms of the set of transcripts by which
they are compatible, but also similar in terms of the conditional probabilities to those
set of transcripts. We observed that this improved factorization leads to improved accuracy of 
the abundance estimation with almost no effect on the speed of the lightweight methods.

In~\cref{chapt5}, we have tackled the problem of estimating the posterior distribution for
RNA-seq abundance estimations. The posterior distribution is necessary to assess the accuracy
of the quantification results. \boots is a popular approach for creating the inferential replicates
from the original observed sample. However, the existing \boot approaches only consider the observed
set of the reads (equivalence classes) for generating the inferential replicates. we introduced the 
concept of \aboots which augments the original sample with additional observations. We illustrated 
how augmenting the \boot samples with reads uniquely mapping to each possible transcript, i.e., the 
set of transcripts with at least one aligned read from the observed sample, improves the uncertainty
estimation.

\subsection{Future Work}

Most RNA-seq quantification tools rely on the generative model proposed by~\cite{Li2010RSEM} about
which we have discussed in detail throughout this Dissertation. A shortcoming of this model is that
it assumes the set of reference sequences against which we are quantifying is complete and all the 
fragments in the sample come from one of the transcripts in the reference. RSEM~\cite{Li2010RSEM} tries to 
address this issue by including an extra transcript as the noise transcript to which all the fragments
not aligned to any other transcripts map. However, based on how the reference transcripts
are created by alternative splicing there might exist a fragment that is completely compatible with
an existing transcript in the reference sequence but it comes from a transcript missed in our
reference transcriptome. This might happen because the exon that the fragment comes from is present
in one of the existing transcripts. Such fragments possibly lead to coverage anomalies of the transcripts
present in the reference which is completely ignored by the generative model of RNA-seq quantification.

Recently, different post quantification analyses have been proposed to detect and possibly fix such
coverage anomalies. E.g., In a recent paper~\cite{ma2019detecting}, the authors develop a method for 
detecting anomalies in the coverages of transcripts in the RNA-seq quantification results. As they state, 
such anomalies could be evidence of misquantification of transcripts. They try to resolve the anomalies by 
moving around reads between transcripts which they map to, in a way that results in less expression anomalies 
while still maintaining a high likelihood. This study also suggests that the cases where the anomalies could 
not be resolved by simply transferring reads between the existing transcripts could indicate some mis-annotated 
transcripts in the reference transcriptome.

% TODO
% Furthermore, the JCC paper.

We believe this issue could be addressed more properly during the quantification step rather than post 
analysis of the abundance estimations.
% proposal to tackle this problem is to detect the transcripts which have significant coverage anomalies, 
% then, attempt re-distributing the reads assigned to those transcripts to other compatible transcripts in a way 
% that reduces the overall coverages anomalies. 
It is possible to detect coverage anomalies during both the online 
and offline phases of the quantification procedure. During the online phase, reads are assigned only once, 
therefore, scanning all the transcripts once at the end of the online phase could highlight the anomalies. However, 
the reads are redistributed during each iteration of the offline EM. Therefore, keeping track of the anomalies, in that case, is more challenging. Another challenge is that the mapping positions are not available in this phase 
in the current implementation of offline EM. One possible solution is keeping track of mapping positions by 
increasing the resolution of the equivalence classes. In addition to these challenges, some anomalies will 
never be resolved by re-distributing the reads, probably because the reads are not sequenced from any of 
the sequences in the reference set. This highlights the existence of novel isoforms which are missing from 
the set of known reference sequences. Therefore, the next step in this would be detecting such cases and 
reporting possible existing transcripts in the reference at the end of the quantification procedure.

